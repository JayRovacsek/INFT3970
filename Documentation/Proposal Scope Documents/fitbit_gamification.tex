\documentclass{article}
    \usepackage{url}
    \usepackage[margin=1.0in]{geometry}
    \usepackage{cite}
    \usepackage{amssymb}
    \usepackage{enumerate}
    \usepackage{enumitem}
    \usepackage{graphicx}
    \usepackage{xcolor}
    \usepackage{pdfpages}
    \usepackage{hyperref}
    \usepackage{listings}
    \usepackage{fancybox}
    \usepackage{lstautogobble}
    \usepackage{titling}
    \usepackage{pdflscape}
    \usepackage[nottoc,notlot,notlof]{tocbibind}
    \renewcommand\maketitlehooka{\null\mbox{}\vfill}
    \renewcommand\maketitlehookd{\vfill\null}

    \graphicspath{ {Resources/} }
    \title{INFT3970 Major Project Scope Document \protect\\
    Gamification of exercise using Fitbit data}
    \author{
        Jay Rovacsek
        \texttt{c3146220@uon.edu.au}
    }
    \date{\today}
    \hypersetup{
    colorlinks=true,
    linkcolor=black,
    filecolor=magenta,      
    urlcolor=blue,
    citecolor=red,
    linktoc=section,
    }
    \pagenumbering{arabic}

    \begin{document}

    \begin{titlingpage}
        \maketitle
    \end{titlingpage}
    \newpage
    \tableofcontents
    \newpage
    
    \section{Project Scope Document}
        \subsection{Introduction}
        The basic concept of this project would be to create a program that gamifies the use of 
        devices such as the fitbit and promotes exercise while supplying useful datapoints to 
        researchers for insight into habbits and trends common in individuals who partake in 
        semi-regular or regular exercise. 
        \subsection{Target Audience}
        The proposed target audience for this project is any individual whom is interested in using
        a fitbit while also exercising a level of casual to competitive nature in the sense of 
        gaming or competing against fellow users of the device.
        \par
        Further information will be required from Geoff as to the design required and limitations of 
        the project, otherwise it is assumed that mostly free-reign over the project will be allowed.
        \subsection{Metrics}
        Proposed metrics to be monitored include:
        \begin{itemize}
            \item Step metrics, with a view to use;
            \begin{itemize}
                \item Average step distance (subject to data available)
            \end{itemize}
            \item Heart-beat metrics, with a view to use;
            \begin{itemize}
                \item Lowest and Highest recorded BPM
                \item Average BPM
            \end{itemize}
            \item Both Step and Heartbeat to include;
            \begin{itemize}
                \item Per Minute
                \item Per Hour
                \item Per Day
                \item Per Week
                \item Per Month
            \end{itemize}
            \item Other metrics to include:
            \begin{itemize}
                \item Continuous days played
                \item Distance travelled (Cap to be discussed/determined)
                \item Greatest North/South movement pattern
                \item Greatest East/West movement pattern
                \item Greatest Spread of location data recorded for various periods
                \item Achievements via Country visit?
            \end{itemize}
        \end{itemize}
        Further metrics could be measured, based on ability to access data such as geolocation,
        sleep data, activity of user social interactions and potentially more.

        \pagebreak

        \subsection{Project Infastructure}
        Proposed hardware for the project would require a number of fitbit, failing the ability to
        locate a number of fitbit, we could script data generation based on expected data ranges and 
        a number of example data structures to mimic the API generated data.
        \par
        The required framework to create the application in is Unity, this allows us to assume we can 
        develop mostly on the Windows platform, use of the mono framework\cite{Mono} would allow us to generate
        an application that could run natively on OSX/UNIX like systems also.
        \\
        Back-end requirements would include a database to store data related to an account, this is 
        proposed to be a single instance that would run global collection of the data and allow access
        only to data on a per user basis. This furthermore allows data-mining of the accounts used,
        allowing generation of both better services and benifit to service provider in the form of
        data for analytics.

        \subsection{Considerations of game genres}
        Game genres that the project could cover include a large number of potential candidate genres, however
        for simplicity of the project and to ease the issue of paradox of choice\cite{Paradox-Of-Choice}
        \\
        A suggested shortlist of game genres that would be well suited to this project include;
        \begin{itemize}
            \item Platformer
            \item RPG
            \item Racing
            \item Horror
        \end{itemize}

        \subsubsection{Platformer}
        A platformer could include the above metrics leading to skewed drop rate of select items from 
        "lootable" objects, decreased enemy count and increased control over player character.
        \\
        A platformer would prove to be a safe game to go with, a "retro" themed approach would work well 
        as we have no digital media majors within the group but are certain we could manage asset generation
        at a basic level.

        \subsubsection{RPG}
        An RPG could include all of the Platformer features and easily extend to include features such as;
        \begin{itemize}
            \item Extra lives
            \item Fast travel abilities
            \item Increased player power
            \item Daily rewards or randomised loot from select tiers of metrics.
        \end{itemize}

        \subsubsection{Racing}
        A Racing game could include usable boots, bonuses and ability to unlock higher tiers of vehicle 
        based on either overall progress in metrics or some alike mechanic.

        \subsubsection{Horror}
        A Horror game could avoid repeated user play requirements by instead requiring the user to wear the fitbit
        while participating in the game. Modifiers to the game could include increased level brightness for decreased 
        heart-rate or a number of interesting mechanics that would be best scoped out after the project is locked into
        choice by all members.
        
        \subsection{Project Services}
            \subsubsection{Database Service}
            A simple RESTful API, data collection on game time played, user choices, user data and all of the 
            associated data that would be collected from the fitbit would be stored by this service.

            \subsubsection{User Management Service}
            A user management serice would be required to allow players to have repeatable or incrementally
            improvable interactions with the solution, data analytics would also be applicable against users
            who opt in/out of the service accordingly after some level of anonymization (or is it anonymisation?) of data.

            \subsubsection{Logging Service}
            The logging service would need to act as middleware between the devices and the database and 
            potentially require some level of parsing of data and verification of data. Failing either of the
            parsing or authentication, the Logging service should push errors to the Error Management service.
            \par
            Without errors occuring, the logging service should pass data required to the database service which
            would perform required tasks to store such data.

            \subsubsection{Error Handling Service}
            The Error handling service would log errors from either data pushed by end devices or the websystem itself,
            the logs would include stack-trace material, cause for logging, associated user data and timestamps.

            \subsubsection{Web Service}
            The web service should include all elements required to report a user-account's associated data
            for selected periods. This should also allow the update of an account, or deletion of an account.
            Included requiremets of this service will include but are not limited to:
            \begin{itemize}
                \item Adding or Removing a device from an account
                \item Visualisation of data associated with an account
                \item Login / Logout functionailty
                \item Downloadable payload of data associated with an account in various formats (XML/json/csv)?
                \item Game status of user
            \end{itemize}
            
    \begin{thebibliography}
        \raggedright
        \bibitem{Paradox-Of-Choice}
            \url{https://books.google.com/books?id=zutxr7rGc_QC&vq=barry+schwartz+paradox+contents}
        \bibitem{Mono}
            \url{https://www.mono-project.com/}
    \end{thebibliography}

    \end{document}  
