\documentclass{article}
    \usepackage{url}
    \usepackage[margin=1.0in]{geometry}
    \usepackage{cite}
    \usepackage{amssymb}
    \usepackage{enumerate}
    \usepackage{enumitem}
    \usepackage{graphicx}
    \usepackage{xcolor}
    \usepackage{pdfpages}
    \usepackage{hyperref}
    \usepackage{listings}
    \usepackage{fancybox}
    \usepackage{lstautogobble}
    \usepackage{titling}
    \usepackage{pdflscape}
    \usepackage[nottoc,notlot,notlof]{tocbibind}
    \renewcommand\maketitlehooka{\null\mbox{}\vfill}
    \renewcommand\maketitlehookd{\vfill\null}

    \graphicspath{ {Resources/} }
    \title{INFT3970 Major Project Scope Document \protect\\
    Distributed Monitoring System using Embedded Devices}
    \author{
        Jay Rovacsek
        \texttt{c3146220@uon.edu.au}\\
        Dean Morton
        \texttt{c3252227@uon.edu.au}
    }
    \date{\today}
    \hypersetup{
    colorlinks=true,
    linkcolor=black,
    filecolor=magenta,      
    urlcolor=blue,
    citecolor=red,
    linktoc=section,
    }
    \pagenumbering{arabic}

    \begin{document}

    \begin{titlingpage}
        \maketitle
    \end{titlingpage}
    \newpage
    \tableofcontents
    \newpage
    
    \section{Project Scope Document}
        \subsection{Introduction}
        The basic concept of this project will be to create a distributed system in which small 
        devices are used to monitor, log and warn of select metrics from a multitude of 
        potential data points.
        \subsection{Target Audience}
        The proposed target audience for this project are a wide range of individuals whom may
        wish to monitor any of the proposed metrics from the wireless devices. The design
        of this system allows for highly customisable systems that would be easy to imlement and 
        preconfigured, with a view to write a management software that would allow end users to
        reconfigure the system as required at any point.
        \subsection{Metrics}
        Proposed metrics to be monitored include:
        \begin{itemize}
            \item Temperature
            \item Pressure (atmospheric)
            \item Humidity
            \item Sound (Decibels)
            \item Sound (Ultrasonic or Subsonic levels)
            \item Wind Speed
            \item UV 
            \item Infared (Movement detection)
            \item Visible Light Levels
            \item Soil Moisture
            \item Acceleration
            \item Orientation/Tilt (x, y, z) of devices
            \item Rain Sensor
            \item Vibration
            \item CO\textsuperscript{2} levels
        \end{itemize}
        Further metrics could be measured, based on availability of sensors, however due to 
        the scope requirements of the course, only a number of the above options will be considered
        for final presentation when the project is complete.

        \pagebreak

        \subsection{Project Infastructure}
        Proposed hardware for the project would run on a mixture of ESP8266\cite{ESP8266} or
        RaspberryPi like boards, made highly available of recent for reasonable prices.
        \\
        Back-end hardware is proposed as any system that can host a number of services related
        to the project. The most essential of these services would include the storage of logging
        achieved by a database. Given the nature of open-source databases we can confidently assume either
        cloud hosting or localised hosting of a system is a non-concern.

        \subsubsection{ESP8266 as the proposed device}
        The ESP8266\cite{ESP8266} would appear to be a highly suitable board fro the project, having
        a number of off-spin boards created based on the original schematic, a high avilability of the
        device and low barriers to entry with the board being programmable from and Windows, OSX or UNIX-like
        host that can interface with USB type B.
        \par
        IDE is completely open to interpretation for the device, only requiring the ability for the IDE to flash
        the device's memory of USB-to-serial-tty or compilabile assets can be directly flashed with utilities such
        as GNU's dd utility or a number of utilities that are both open source and free such as NodeMCUFlasher\cite{NodeMCU}, or
        ArduinoIDE\cite{ArduinoIDE} of which, all are cross platform (dd available on windows hosts via Cygwin).
        \par
        The ESP8266\cite{ESP8266} is supported heavily with development within the 'maker' world, the extensibility
        of such devices are only limited by existence of peripheral devices in which to interface with.
        A large number of tutorials, how-to's and professionally compiled examples exist also\cite{ESP8266-How-To}

        \subsubsection{Raspberry Pi as th proposed device}
        The Raspberry Pi\cite{RaspberryPi} also would suit as a very capable board which is extremely well supported 
        by the 'maker' space. Benefits of the Pi over the ESP are an ability to develop in higher level languages
        such as Python, Bash and Java. The Pi also boasts a processing power in the range of 10-20x more powerful than the 
        ESP, uses a quad core processor and can support onboard GPU processing.
        \\
        The RaspberryPi also opens us a wide range of preprocessing and device side validation given the 
        offering of utilities and power of the module. 
        \par
        However the RaspberryPi is significantly more expensive, between 2-3x larger than the ESP in physical form
        and would be much harder to setup under a battery style mode in comparison to the ESP8266.

        \subsection{Project Services}
            \subsubsection{Database Service}
            The database service would be a simple RESTful API.

            \subsubsection{User Management Service}
            A user management serice would be required to ensure security of an individuals privacy.
            This, as above would be best preflashed and not apparent to the end-user, taking the form of
            some kind of expirable token.
            \\
            The service would require some kind of method to associate a user with devices at the point of sale
            or some kind of method akin to that implemented by the Particle board range \cite{Particle}
            which associates a serial or UUID/GUID code to identify the device in question.
            This would be pre-programmed at build time to be stored in the database so no management 
            other than the entry of such a UUID/GUID would be required by the end-user.

            \subsubsection{Logging Service}
            The logging service would need to act as middleware between the devices and the database and 
            potentially require some level of parsing of data and verification of data. Failing either of the
            parsing or authentication, the Logging service should push errors to the Error Management service.
            \par
            Without errors occuring, the logging service should pass data required to the database service which
            would perform required tasks to store such data.

            \subsubsection{Error Handling Service}
            The Error handling service would log errors from either data pushed by end devices or the websystem itself,
            the logs would include stack-trace material, cause for logging, associated user data and timestamps.

            \subsubsection{Web Service}
            The web service should include all elements required to report a user-account's associated data
            for selected periods. This should also allow the update of an account, or deletion of an account.
            Included requiremets of this service will include but are not limited to:
            \begin{itemize}
                \item Adding or Removing a device from an account
                \item Visualisation of data associated with an account
                \item Login / Logout functionailty
                \item Downloadable payload of data associated with an account in various formats (XML/json/csv)?
            \end{itemize}
            \par
            The proposed framework to complete the web systems required for this project are one of the following 
            candidates: Angular MVC application\cite{Angular}, React Native\cite{ReactNative} or ASP.NETCore MVC 
            application\cite{ASP.NETcore-MVC}.
            The discussion of frameworks should be followed up in th following week, and a candidate chosen
            by the team in consensus.
            
    \newpage

    \begin{thebibliography}
        \raggedright
        \bibitem{Particle}
            \url{https://www.particle.io/}
        \bibitem{ESP8266-How-To}
            \url{http://esp8266.net/}
        \bibitem{ESP8266}
            \url{https://randomnerdtutorials.com/home-automation-using-esp8266/}
        \bibitem{NodeMCU}
            \url{https://github.com/nodemcu/nodemcu-flasher}
        \bibitem{ArduinoIDE}
            \url{https://create.arduino.cc/editor}
        \bibitem{RaspberryPi}
            \url{https://www.raspberrypi.org/}
            \bibitem{ReactNative}
            \url{https://facebook.github.io/react-native/}
            \bibitem{Angular}
            \url{https://www.nativescript.org/nativescript-is-how-you-build-native-mobile-apps-with-angular}
            \bibitem{ASP.NETcore-MVC}
            \url{https://docs.microsoft.com/en-us/aspnet/core/tutorials/first-mvc-app/?view=aspnetcore-2.1}
    \end{thebibliography}

    \end{document}  
