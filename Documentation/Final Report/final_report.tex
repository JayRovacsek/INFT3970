\documentclass[a4paper,12pt,headings=normal]{article}
    \usepackage{url}
    \usepackage{xcolor}
    \usepackage{lscape}
    \usepackage{amssymb}
    \usepackage{titling}
    \usepackage{pdfpages}
    \usepackage{enumitem}
    \usepackage{graphicx}
    \usepackage{hyperref}
    \usepackage{listings}
    \usepackage{fancybox}
    \usepackage{enumerate}
    \usepackage{pdflscape}
    \usepackage{afterpage}
    \usepackage{lstautogobble}
    \usepackage[utf8]{inputenc}
    \usepackage[english]{babel}
    \usepackage[toc,page]{appendix}
    \usepackage{csquotes}
    \usepackage{comment}
    \usepackage[normalem]{ulem}
    \usepackage[margin=0.8in]{geometry}
    \usepackage[nottoc,notlot,notlof]{tocbibind}
    \usepackage[
        backend=biber,
        style=ieee,
        citestyle=ieee
        ]{biblatex}

    \addbibresource{report.bib}

    \DeclareFixedFont{\ttb}{T1}{txtt}{bx}{n}{12} % for bold
    \DeclareFixedFont{\ttm}{T1}{txtt}{m}{n}{12}  % for normal

    \definecolor{deepblue}{rgb}{0,0,0.5}
    \definecolor{deepred}{rgb}{0.6,0,0}
    \definecolor{deepgreen}{rgb}{0,0.5,0}

    \newcommand\gostyle{\lstset{
        language=Go,
        autogobble=true,
        basicstyle=\small,
        stepnumber=1,
        keywordstyle=\small\color{deepblue},
        emph={
            fmt,
            gobot,
            main,
            tryFlashLED,
            flashLED,
            getAvailability,
            getSensorData,
            printError
            },
        emphstyle=\small\color{deepred},
        numbers=left,
        tabsize=4,
        stringstyle=\color{deepgreen},
        breaklines=true,
        frame=tb,
        showstringspaces=false
        }}

        % Go environment
        \lstnewenvironment{Go}[1][]
        {
        \gostyle
        \lstset{#1}
        }
        {}
    
        % Go for external files
        \newcommand\gostyleexternal[2][]{{
            \gostyle
            \lstinputlisting[#1]{#2}}}

    \renewcommand\maketitlehookd{\vfill\null}
    \renewcommand\maketitlehooka{\null\mbox{}\vfill}

    \newcounter{num}

    \graphicspath{ {Images/} }

    \title{INFT3970 Major Project Final Report \protect\\
    Distributed Monitoring System using ESP-12E}
    \author{
        Team Encore\\
        \textit{Thursday 10:00AM - 10:30AM}\\
        Jay Rovacsek
        \texttt{c3146220@uon.edu.au}\\
        Dean Morton
        \texttt{c3252227@uon.edu.au}\\
        Josh Brown
        \texttt{c3283797@uon.edu.au}\\
        Jacob Litherland
        \texttt{c3263482@uon.edu.au}\\
        Lee Marron
        \texttt{c3263482@uon.edu.au}\\
        Edward Lonsdale
        \texttt{c3252144@uon.edu.au}
    }
    \date{\today}
    \hypersetup{
    colorlinks=true,
    linkcolor=black,
    filecolor=magenta,      
    urlcolor=blue,
    citecolor=red,
    linktoc=section,
    }
    \pagenumbering{arabic}

    \newlist{legal}{enumerate}{10}
    \setlist[legal]{label*=\arabic*.}

    \begin{document}

    \begin{titlingpage}
        \maketitle
    \end{titlingpage}

    \tableofcontents
    
    \newpage

    \section{Executive Summary}

    \newpage
    \section{Introduction}
        \subsection{Project Objectives}
        \subsection{Business Summary}

    \newpage
    \section{Proposed Solution}
        \subsection{Outline}
        \subsection{Technical Requirements}
        \subsection{Limitations or Constraints}

    \newpage
    \section{Technical Documentation}
        \subsection{Database Documentation}
            \subsubsection{Relational Model}
            \subsubsection{Data Requirements}
            \subsubsection{Data Dictionary}
            \subsubsection{Database Schema}

        \newpage
        \subsection{ESP12-E Documentation}
            \subsubsection{Temperature}
            \subsubsection{Humidity}
            \subsubsection{Motion}
            \subsubsection{Issues in Development}
                One of the first challenges of this project was to ensure we had a viable
                path to producing a result we deemed desirable, a rused prototype was created
                on the ESP12-E using arduino code.
                The code used methods causing blocks on continued execution and for one or 
                two items being reported this was okay, however to ensure the platform had
                room to breathe we investigated two routes: non-blocking code using timers 
                in arduino code, and golang utilising the Gobot framework\cite{GoBot}.

                \medskip

                The Gobot route seemed very promising, prewritten libraries for items 
                such as the XC-4444\cite{XC-4444} existed, however a mixture of team members 
                unfamiliar with the Go langauge\cite{Golang} and a number of resources
                pointing to the DHT-11\cite{DHT11} being severely undersupported\cite{GolangDHT111Issues}.
                in the Golang space lead us to investigate the alternative option: timers 
                in arduino code.

                A copy of the Golang Arduino Implementation can be found \hyperref[sec:GolangESP12EImplementation]{\color{blue}here.}

                \medskip


        \newpage
        \subsection{Web Application Documentation}
            \subsubsection{API Documentation}
            \subsubsection{Issues in Development}

        \newpage
        \subsection{General Documentation}
            \subsubsection{Sequence Diagrams}
            \subsubsection{Network Diagram}

        \newpage
        \subsection{Install Proceedures}
    
    \newpage
    \section{User Documentation}
        \subsection{Installation}
        \subsection{Accessing Dashboards}
        \subsection{Account Administration}

    \newpage
    \printbibliography

    \newpage
    \appendix
    \section{Appendix}
    \subsection{Golang ESP12-E Implementation}
    \label{sec:GolangESP12EImplementation}
    \gostyleexternal{Resources/test.go}

    \end{document}
